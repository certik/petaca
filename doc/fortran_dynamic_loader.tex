\documentclass[11pt]{article}

\usepackage{lmodern}
\usepackage[T1]{fontenc}

\usepackage{underscore}
\usepackage{verbatim}
\usepackage{enumitem}

\usepackage[nofancy]{latex2man}

\usepackage[margin=1.25in,letterpaper]{geometry}

%\setlength{\parindent}{0pt}

\begin{document}

\setDate{June 2013}
\setVersion{1.0}

\begin{Name}{3}{map_any_type}{Neil N. Carlson}{Petaca}{The fortran_dynamic_loader module}
The \texttt{fortran_dynamic_loader} module defines an object-oriented Fortran
interface to the system dynamic loader as implemented by the POSIX C functions
\emph{dlopen}, \emph{dlclose}, \emph{dlsym}, and \emph{dlerror}.
\end{Name}

\section{Synopsis}
\begin{description}[style=nextline]\raggedright
\item[Usage]
  \verb+use fortran_dynamic_loader+
\item[Derived Type]
  \texttt{shlib}
\item[Parameters]
  \texttt{RTLD_LAZY},\texttt{ RTLD_NOW},\texttt{ RTLD_LOCAL},%
  \texttt{ RTLD_GLOBAL}
\item[Linking]
  Link with the system DL library (\texttt{-ldl} on Linux) to
  resolve the symbols \emph{dlopen}, \emph{dlclose}, \emph{dlsym}, and
  \emph{dlerror}.
\end{description}

\section{The shlib derived type}
The derived type \texttt{shlib} implements the dynamic loading of a shared
library and access to data and procedures defined by the library.

\subsection{Type bound subroutines}
The derived type has the following type bound subroutines.
Each subroutine has the optional intent-out arguments \texttt{stat} and
\texttt{errmsg}.  If the integer \texttt{stat} is present, it is assigned
the value 0 if the subroutine completes successfully, and a nonzero value
if an error occurs.  In the latter case, the allocatable character string
\texttt{errmsg}, if present, is assigned the error string returned by the
underlying system dl library.  If \texttt{stat} is not present and an error
occurs, the error string is written to the preconnected error unit and the
program exits with a nonzero status.
\begin{description}[style=nextline]\setlength{\itemsep}{0pt}
\item[\texttt{open(filename, mode [,stat [,errmsg]])}]
  loads the shared library file named by the character argument
  \texttt{filename} and associates it with the \texttt{shlib} object.
  If \texttt{filename} contains a slash (\texttt{/}), then it is interpreted
  as a relative or absolute pathname.  Otherwise the dynamic loader searches
  a certain list of directories for the library; see \Cmd{dlopen}{3} for a
  detailed description of the search process.
  
  One of the following two values must be passed as the \texttt{mode} argument:
  \begin{description}[style=nextline]\setlength{\itemsep}{0pt}
  \item[\texttt{RTLD_LAZY}]
    Only resolve symbols as the code that references them is executed
    (lazy binding).
  \item[\texttt{RTLD_NOW}]
    All undefined symbols in the library are resolved before the \texttt{open}
    procedure returns.  An error occurs if this is not possible.  This
    is also the behavior if the environment variable \texttt{LD_BIND_NOW} is
    set to a nonempty string.
  \end{description}
  One of the following values may optionally be or'ed with the preceding values
  before being passed as the \texttt{mode} argument; e.g.,
  \verb+mode=ior(RTLD_LAZY,RTLD_GLOBAL)+.
  \begin{description}[style=nextline]\setlength{\itemsep}{0pt}
  \item[\texttt{RTLD_GLOBAL}]
    The symbols defined by this library will be made available for symbol
    resolution of subsequently loaded libraries.
  \item[\texttt{RTLD_LOCAL}]
    This is the converse of \texttt{RTLD_GLOBAL} and the default. Symbols
    defined by this library are not made available to resolve references
    in subsequently loaded libraries.
  \end{description}
  See \Cmd{dlopen}{3} for more details.
\item[\texttt{close([stat [,errmsg]])}]
  decrements the reference count on the shared library.  When the reference
  count reaches zero, the shared library is unloaded.  See \Cmd{dlclose}{3}
  for a detailed description of the behavior.
\item[\texttt{func(symbol, funptr [,stat [,errmsg]])}]
  returns the memory address where the specified function symbol from the
  shared library is loaded.  The character argument \texttt{symbol} gives
  the symbol name, and the address is returned in the \texttt{type(c_funptr)}
  argument \texttt{funptr}.  The caller is responsible for converting this
  C function pointer value to an appropriate Fortran procedure pointer using
  the subroutine \texttt{c_f_procpointer} from the intrinsic
  \texttt{iso_c_binding} module.
\item[\texttt{sym(symbol, symptr [,stat [,errmsg]])}]
  returns the memory address where the specified data symbol from the shared
  library is loaded.  The character argument \texttt{symbol} gives the symbol
  name, and the address is returned in the \texttt{type(c_ptr)} argument
  \texttt{symptr}.  The caller is responsible for converting this C pointer
  value to an appropriate Fortran data pointer using the subroutine
  \texttt{c_f_pointer} from the intrinsic \texttt{iso_c_binding} module.
\end{description}

\section{Example}
\begin{verbatim}
use fortran_dynamic_loader
use,intrinsic :: iso_c_binding, only: c_funptr, c_f_procpointer

abstract interface
  real function f(x)
    real, value :: x
  end function
end interface
procedure(f), pointer :: cbrtf

type(shlib) :: libm
type(c_funptr) :: funptr

!! Load the C math library libm.so and calculate the cube
!! root of 8 using the function cbrtf from the library.
call libm%open ('libm.so', RTLD_NOW)
call libm%func ('cbrtf', funptr)
call c_f_procpointer (funptr, cbrtf)
if (cbrtf(8.0) /= 2.0) print *, 'error'
call libm%close
\end{verbatim}

\section{Bugs}
Bug reports and improvement suggestions should be directed to
\Email{neil.n.carlson@gmail.com}

\LatexManEnd

\end{document}
